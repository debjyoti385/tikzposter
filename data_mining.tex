%%
%% This is file `tikzposter-template.tex',
%% generated with the docstrip utility.
%%
%% The original source files were:
%%
%% tikzposter.dtx  (with options: `tikzposter-template.tex')
%% 
%% This is a generated file.
%% 
%% Copyright (C) 2014 by Pascal Richter, Elena Botoeva, Richard Barnard, and Dirk Surmann
%% 
%% This file may be distributed and/or modified under the
%% conditions of the LaTeX Project Public License, either
%% version 2.0 of this license or (at your option) any later
%% version. The latest version of this license is in:
%% 
%% http://www.latex-project.org/lppl.txt
%% 
%% and version 2.0 or later is part of all distributions of
%% LaTeX version 2013/12/01 or later.
%% 





\documentclass[17pt, a1paper, landscape, margin=0mm, innermargin=15mm,
blockverticalspace=15mm, colspace=15mm, subcolspace=8mm]{tikzposter}
% \documentclass{tikzposter} %Options for format can be included here
% \setcolumnnumber{3}
 % Title, Author, Institute
\title{\textbf{News Clustering and Event Detection}, Data Mining, CS6140, Spring 2016}
\author{\textbf{Debjyoti Paul}, deb@cs.utah.edu, \textbf{Yogesh Mishra}, ymishra@cs.utah.edu }
\institute{\textbf{School of Computing, University of Utah}}
\titlegraphic{
\includegraphics[width=0.14\textwidth]{figures/ulogo.png}
}

\usepackage{enumitem}
\usepackage{url}
\usepackage{tabularx}
\usepackage{mathtools}
\usepackage{blindtext}
%%%%%%%%% Awesome tex stack exchange answer 
% http://tex.stackexchange.com/questions/167521/tikz-poster-aligning-titlegraphic-to-the-right-of-the-title
%%%%%%%%%%%%%%%%%%%%%%%%%%%%%%%%%%%%%%%%%%%%%%%%%%%

\makeatletter
\renewcommand\TP@maketitle{%
   \centering
         \tikz[remember picture,overlay]\node[scale=0.8,anchor=east,xshift=0.5\linewidth,yshift=2.3cm,inner sep=0pt] {%
       \@titlegraphic
    };
   \begin{minipage}[b]{0.8\linewidth}
        \centering
        \color{titlefgcolor}
        {\bfseries \Huge \sc \@title \par}
        \vspace*{1em}
        {\huge \@author \par}
        \vspace*{1em}
        {\LARGE \@institute}
    \end{minipage}%
}
\makeatother


%Choose Layout
\usetheme{Basic}
% \usecolorpalette{PurpleGrayBlue}
\usecolorstyle{dmcolor}
\usebackgroundstyle{BottomVerticalGradation}

\usetitlestyle{Envelope}

\begin{document}

\maketitle
\begin{columns}
\column{0.67}
%\note[targetoffsetx=3.2cm, targetoffsety=-10cm, angle=20, rotate=12]
%{
%Since \textbf{Reading Comprehension Test} has a defined domain of search space, our QA system does not require \textbf{Document Retrieval module}. Our focus is mainly on building an efficient \textbf{Answer Processing} module.
%}

\begin{subcolumns}

\subcolumn{0.5}
\block{Introduction}{
%   This projects finds events from news articles, categorize them, extracts relevant information about event. Pure clustering approach is not sufficient to extract events out of huge corpus of news articles.
%   \\[10pt]
  \textbf{Problem Statement} 
  Given a large corpus of mixed heterogeneous articles.  Find very similar articles that represents \emph{Event}s and define their \emph{Topic} and \emph{Reachability}. 
  \begin{center}  
  \textbf{ Corpus $C=\{d_1,d_2,d_3,\ldots,d_n\}$ \\
  Temporal property $T=\{t_{d_1},t_{d_2},t_{d_3}, \ldots,t_{d_n}\}$. } 
  \end{center}
  \vspace{0.2cm}
\textbf{\emph{Event : }} An Event $E$ is described as a set of very similar documents   having a near/close temporal property and describing same context.\\
$$E=\{d_1,d_2,...d_m\}, ~~ \forall i,j ~~ \texttt{sim}(d_i,d_j) \geq \tau
,~~ \forall i,j ~~ | t_{d_i} - t_{d_j}| \leq \nu $$ 
\hfill where $\tau$ is similarity threshold and $\nu$ is temporal locality constant usually in days.\\
 }




% { % NEW NOTE STYLE
% \colorlet{notebgcolor}{white!10!white}




\note[targetoffsetx=-1cm, targetoffsety=-4.5cm, angle=-90, width=0.28\textwidth, rotate=0,]
{

\textbf{A Similarity Measures for Text Classification and Clustering} %\cite{6420834} 
\\
$$\texttt{Sim}(x,y) = \frac{F(x,y) +\lambda}{1 + \lambda}$$
$$F(x,y)= \frac{\sum _{j=}^m N^* (x_j,y_j)}{\sum _{j=}^m N_{\cup} (x_j,y_j)} $$
\[
    N^*(x,y)= 
\begin{dcases}
    0.5 \Big(1+ exp \{- \big(\frac{x_j-y_j}{\sigma_j}\big)^2\}\Big),& \text{if } x_jy_j > 0\\
    0,              & \text{if } x_j=0 \text{ and } y_j=0\\
    -\lambda       & \text{otherwise}
\end{dcases}
\]
\[
    N_{\cup}(x,y)= 
\begin{dcases}
    0,&  \text{if } x_j=0 \text{ or } y_j=0\\
    1      & \text{otherwise}
\end{dcases}
\]
\hfill where $\lambda$ is any constant with $0< \lambda \leq 1 $ and $\sigma_j$ is the variance of word $j$ in the corpus. \\



}

% }


\subcolumn{0.5}
%  \block[bodywidthscale=1.05,bodyinnersep=4em]{Dataset}{
 \block{Dataset}{
         % \textbf{How we obtained our data?}\\
%  Generic Crawler 12 main Indian news sites. 
%  We developed a crawler to collect news articles, clean it and store them into a database for faster/easier access. Features of Dataset and Crawler Module:\\

\begin{enumerate}[noitemsep,nolistsep]
\item[] \textbf{Generic Crawler} framework to crawl articles from news sites. Extensible and customized storage support.
\item[] \textbf{Technologies:} \textit{Python Scrapy}, \textit{Splash} bundled with \textit{docker}
\item[] \textbf{Data Cleansing:} Discard malformed data and store articles in proper json schema.
\item[] \textbf{Dataset:} \textbf{5 months} (Nov 2015 - Mar 2016), \textbf{12} news sites, \textbf{5,24,000+} articles, \textbf{2.4 GB} of articles, \textbf{3000 articles} per day. 
\end{enumerate}
}

{ % NEW NOTE STYLE
\colorlet{notebgcolor}{white!10!white}

\note[targetoffsetx=2.8cm, targetoffsety=1.2cm, angle=-105, width=0.28\textwidth, rotate=0]
{
\textbf{News Sites Crawled:}\\ 
1. The Better India \quad ~ 2. The Hindu \quad ~ 3. Times of India  \quad ~~~~~ 4. Economic Times  \\
\quad  5. Financial Express \quad 6.The Statesman  \quad 7. Business Standard \quad 8. Assam Tribune\\
\quad 9. NDTV  \quad \quad~~~~~~~~~ 10. Daily Excelsior \quad ~~ 11. First Post \quad ~~~~ 12. India Today
}

}

\end{subcolumns}



\begin{subcolumns}
\subcolumn{0.42}

\block{}{
        \vspace{10.5cm}
        \textbf{\Large{Phrase LDA Topic Modeling:}} %\textit{(Properties)} \\
\\
Discovers \textbf{fine grained topics} from titles, focus of articles.\\ 
\textbf{(i) Frequency:} Relays topic information.\\
\textbf{(ii) Collocation:} ``Strong tea'' vs ``Powerful tea'' $\rightarrow$ statistical biasedness. \\
\textbf{(iii) Completeness:} Subset phrases should comply with (i) and (ii) \\
\vspace{4cm}\\
\textbf{\large{Findings:}}\\
Phrase contributing to a topic changes with time. \\
Phrase LDA over two months data decreases effectiveness of phrase contributing to a topic and adds noise. \\
Phrase LDA on one month of data produces optimal results. \\
50-75 fine grained topics every run. \\
}
{ % NEW NOTE STYLE
\colorlet{notebgcolor}{white!10!white}

\note[targetoffsetx=-3cm, targetoffsety=-3.5cm, angle=-20, width=0.25\textwidth, rotate=0, linewidth=0pt, connection]
{
                \textbf{\Large{Fine Grained Topic  $\rightarrow$  Broad Categorization}}\\
                \begin{minipage}[t]{0.22\textwidth}
                        \flushright
                (Mutually exclusive categories)
                \end{minipage}
                \\
\textit{E.g.} Finance, Security and Terrorism, Sports, Entertainment etc.
}

\note[targetoffsetx=21cm, targetoffsety=-0.8cm, angle=0, width=0.35\textwidth, rotate=0, roundedcorners=5, linewidth=0pt]
{

        \begin{center}
        \textbf{\Large{Temporal span~~ $|$ ~~Sliding Window ~~$|$~~  Intra-Category} }
%\begin{table}[H]
%\label{broad_category}
\begin{tabularx}{0.30\textwidth}{XXX} 
~~~~~An event usually has &  ~~~~~~ Articles can merge to  &  ~~~~~~~~~~~~Related events  \\
~~~3 days of temporal locality & ~~~~previously detected events  &  ~~~~~~~~~under one category \\
\end{tabularx}
%\end{table}
\end{center}

}



}

\subcolumn{0.65}
\block{Our Methods}{
        \textbf{\Large{Document Representation}}  \\
        Modified TF-IDF sparse vector with 3 schemes. \\
        \vspace{-3cm}
\begin{tikzfigure}[]

\label{fig:fig3}
\includegraphics[width=0.30\textwidth]{figures/document_rep.png}
\end{tikzfigure}

\vspace{7.5cm}

\begin{minipage}[t]{0.25\textwidth}
\end{minipage}
\begin{minipage}[t]{0.34\textwidth}
  \flushright
 \textbf{\Large{Event Detection}}  \\
   % Intra-\textit{Broad Category} clustering \\
   \textbf{Hierarchical clustering:} Applied average linkage clustering \\
   \textbf{Graph clustering:} Finding strongly-connected subgraphs  \\
   \flushleft
   Graph $G$ created with similarity value of documents.\\ 
   A threshold of $\tau$ applied and pruned edges less than $\tau$.\\
   Applied strongly connected subgraph algorithm.\\
   \begin{center}
   \textbf{ 
   $\alpha, \beta, \gamma$- docuent representation with Graph clustering worked best.}
    \end{center}
\end{minipage}

}

{ % NEW NOTE STYLE
\colorlet{notebgcolor}{white!10!white}




%\note[targetoffsetx=-1.2cm, targetoffsety=-15.2cm, angle=90, width=0.25\textwidth, rotate=0, roundedcorners=0, linewidth=0pt]
%{
%
%\textbf{Document Representation:}\\
%\begin{tikzfigure}[ Articles to Event aggregation ]
%\label{fig:fig2}
%\includegraphics[width=0.22\textwidth]{figures/event.png}
%\end{tikzfigure}
%
%
%}
%






}


\note[targetoffsetx=-11.2cm, targetoffsety=-10.2cm, angle=-135, width=0.45\textwidth, rotate=0]
{
\textbf{\emph{Event Reachability : }} Reachability of an event is the spatial region over which the event has its impact. \\
When an Event $E$ has impact on locations $L=\{l_1,l_2,l_3,\ldots,l_p\}$ the convex hull of $L$ will be defined as the reachability of the Event.  
}
\end{subcolumns}
 

\column{0.33}
\block{News Event Detection Architecture}{
\begin{tikzfigure}[Basic Question Answering System ]
\label{fig:fig1}
\includegraphics[width=0.30\textwidth,height=11cm]{figures/EventDetection.png}
\end{tikzfigure}

\vspace{-1cm}

\begin{tikzfigure}[ Articles to Event aggregation ]
\label{fig:fig2}
\includegraphics[width=0.22\textwidth]{figures/event.png}
\end{tikzfigure}


\vspace{1cm}
\textbf{\Large{Modules:}}
\noindent
\begin{itemize}
\item[] \textbf{Crawler: } Crawls 12 News sites and sanitizes data with proper structure. 
\item[] \textbf{News Article Database: } Elasticsearch Cluster running on  \texttt{db03.cs.utah.edu}
\item[] \textbf{Topic Modeling: } Using Phrase LDA to detect Fine grained Categories
\item[] \textbf{Broad Categorization: } Mapping Fine Grained categories to mutual exclusive Broad categories  
\item[] \textbf{Clustering \& Event Detection: } Experimentation with different similarity measure techniques to cluster very similar articles with different approaches, e.g. Graph Algorithms and $k$-Means
\end{itemize}


\textbf{\Large{Citations:}} \textit{(Detailed citations in report)}
\begin{enumerate}[noitemsep,nolistsep]
        \item \textit{El. Kishky}, Scalable topical phrase miningfrom text corpora
        \item \textit{Y. S. Lin et.al.} A similarity measure for text classification and clustering 
        \item \textit{Kumaran. G et.al}, Text classification and named entities for new event detection
        \item \textit{Jiaul H. Paik.}, A novel tf-idf weighting scheme for effective ranking.
\end{enumerate}
}




\end{columns}


%%%%%%%%% Suggestion from 
% http://tex.stackexchange.com/questions/233854/tikz-poster-block-title-colours
%%%%%%%%%%
{
\colorlet{notebgcolor}{colorOne!10!white}

}
%  \block{QA SYSTEM HISTORY}{
%  \textbf{[1968-71]} Terry Winograd, MIT Artificial Intelligence Laboratory developed \textbf{SHRDLU} where computer can answer short block of response to query.\\
%  \textbf{[1978]}  Lehnert\'s thesis, first to proposed a question answering system based on semantics and reasoning.\\
%  \textbf{[1980s]} \textbf{LUNAR} and \textbf{BASEBALL} QA system appears\\
%  \textbf{[1993]} START by Boris Katz, first web based QA system at CSAIL, MIT\\
%  \textbf{[2000]} \textbf{Semantic and rule based QA system} gained attention, QA Roadmap paper published. Contirbution from well known researchers around the globe; including E. Riloff. (Our cs6340 Instructor)\\
%  \textbf{[2003]} \textbf{SIRI} started in 2003 funded by DARPA till 2008. Integrated with apple iOS4 in 2010.\\
%  \textbf{[2007-2011]} \textbf{IBM Watson},  it was a breakthrough in open domain
% Question Answering System.
%  }


\end{document}


% MINIPAGE EXAMPLE
%\begin{minipage}[t]{1in}
%\end{minipage}
%\hfill
%\begin{minipage}[t]{5in}
% 1. Rule Based scores \textbf{ $0 \leq f_1(x) \leq 1 $} \\
% 2. \textbf{0.25} for best matching, \textbf{0.15} for probable and \textbf{0.05} for 'may be' answers.\\
% 3. Scoring function of Similarity Module \\ 
% \hspace*{0.15in} \textbf{$0 \leq f_2(x) \leq 1 $} \\
%\end{minipage}
%\hspace*{1in} 
%4. Hypernyms matched words score is less than the synsets matched words $f_2(hypernym) < f_2(synset)$ \\


\endinput
%%
%% End of file `tikzposter-template.tex'.


